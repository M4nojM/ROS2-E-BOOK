% Options for packages loaded elsewhere
\PassOptionsToPackage{unicode}{hyperref}
\PassOptionsToPackage{hyphens}{url}
\PassOptionsToPackage{dvipsnames,svgnames,x11names}{xcolor}
%
\documentclass[
  letterpaper,
  DIV=11,
  numbers=noendperiod]{scrreprt}

\usepackage{amsmath,amssymb}
\usepackage{lmodern}
\usepackage{iftex}
\ifPDFTeX
  \usepackage[T1]{fontenc}
  \usepackage[utf8]{inputenc}
  \usepackage{textcomp} % provide euro and other symbols
\else % if luatex or xetex
  \usepackage{unicode-math}
  \defaultfontfeatures{Scale=MatchLowercase}
  \defaultfontfeatures[\rmfamily]{Ligatures=TeX,Scale=1}
\fi
% Use upquote if available, for straight quotes in verbatim environments
\IfFileExists{upquote.sty}{\usepackage{upquote}}{}
\IfFileExists{microtype.sty}{% use microtype if available
  \usepackage[]{microtype}
  \UseMicrotypeSet[protrusion]{basicmath} % disable protrusion for tt fonts
}{}
\makeatletter
\@ifundefined{KOMAClassName}{% if non-KOMA class
  \IfFileExists{parskip.sty}{%
    \usepackage{parskip}
  }{% else
    \setlength{\parindent}{0pt}
    \setlength{\parskip}{6pt plus 2pt minus 1pt}}
}{% if KOMA class
  \KOMAoptions{parskip=half}}
\makeatother
\usepackage{xcolor}
\setlength{\emergencystretch}{3em} % prevent overfull lines
\setcounter{secnumdepth}{5}
% Make \paragraph and \subparagraph free-standing
\ifx\paragraph\undefined\else
  \let\oldparagraph\paragraph
  \renewcommand{\paragraph}[1]{\oldparagraph{#1}\mbox{}}
\fi
\ifx\subparagraph\undefined\else
  \let\oldsubparagraph\subparagraph
  \renewcommand{\subparagraph}[1]{\oldsubparagraph{#1}\mbox{}}
\fi

\usepackage{color}
\usepackage{fancyvrb}
\newcommand{\VerbBar}{|}
\newcommand{\VERB}{\Verb[commandchars=\\\{\}]}
\DefineVerbatimEnvironment{Highlighting}{Verbatim}{commandchars=\\\{\}}
% Add ',fontsize=\small' for more characters per line
\usepackage{framed}
\definecolor{shadecolor}{RGB}{241,243,245}
\newenvironment{Shaded}{\begin{snugshade}}{\end{snugshade}}
\newcommand{\AlertTok}[1]{\textcolor[rgb]{0.68,0.00,0.00}{#1}}
\newcommand{\AnnotationTok}[1]{\textcolor[rgb]{0.37,0.37,0.37}{#1}}
\newcommand{\AttributeTok}[1]{\textcolor[rgb]{0.40,0.45,0.13}{#1}}
\newcommand{\BaseNTok}[1]{\textcolor[rgb]{0.68,0.00,0.00}{#1}}
\newcommand{\BuiltInTok}[1]{\textcolor[rgb]{0.00,0.23,0.31}{#1}}
\newcommand{\CharTok}[1]{\textcolor[rgb]{0.13,0.47,0.30}{#1}}
\newcommand{\CommentTok}[1]{\textcolor[rgb]{0.37,0.37,0.37}{#1}}
\newcommand{\CommentVarTok}[1]{\textcolor[rgb]{0.37,0.37,0.37}{\textit{#1}}}
\newcommand{\ConstantTok}[1]{\textcolor[rgb]{0.56,0.35,0.01}{#1}}
\newcommand{\ControlFlowTok}[1]{\textcolor[rgb]{0.00,0.23,0.31}{#1}}
\newcommand{\DataTypeTok}[1]{\textcolor[rgb]{0.68,0.00,0.00}{#1}}
\newcommand{\DecValTok}[1]{\textcolor[rgb]{0.68,0.00,0.00}{#1}}
\newcommand{\DocumentationTok}[1]{\textcolor[rgb]{0.37,0.37,0.37}{\textit{#1}}}
\newcommand{\ErrorTok}[1]{\textcolor[rgb]{0.68,0.00,0.00}{#1}}
\newcommand{\ExtensionTok}[1]{\textcolor[rgb]{0.00,0.23,0.31}{#1}}
\newcommand{\FloatTok}[1]{\textcolor[rgb]{0.68,0.00,0.00}{#1}}
\newcommand{\FunctionTok}[1]{\textcolor[rgb]{0.28,0.35,0.67}{#1}}
\newcommand{\ImportTok}[1]{\textcolor[rgb]{0.00,0.46,0.62}{#1}}
\newcommand{\InformationTok}[1]{\textcolor[rgb]{0.37,0.37,0.37}{#1}}
\newcommand{\KeywordTok}[1]{\textcolor[rgb]{0.00,0.23,0.31}{#1}}
\newcommand{\NormalTok}[1]{\textcolor[rgb]{0.00,0.23,0.31}{#1}}
\newcommand{\OperatorTok}[1]{\textcolor[rgb]{0.37,0.37,0.37}{#1}}
\newcommand{\OtherTok}[1]{\textcolor[rgb]{0.00,0.23,0.31}{#1}}
\newcommand{\PreprocessorTok}[1]{\textcolor[rgb]{0.68,0.00,0.00}{#1}}
\newcommand{\RegionMarkerTok}[1]{\textcolor[rgb]{0.00,0.23,0.31}{#1}}
\newcommand{\SpecialCharTok}[1]{\textcolor[rgb]{0.37,0.37,0.37}{#1}}
\newcommand{\SpecialStringTok}[1]{\textcolor[rgb]{0.13,0.47,0.30}{#1}}
\newcommand{\StringTok}[1]{\textcolor[rgb]{0.13,0.47,0.30}{#1}}
\newcommand{\VariableTok}[1]{\textcolor[rgb]{0.07,0.07,0.07}{#1}}
\newcommand{\VerbatimStringTok}[1]{\textcolor[rgb]{0.13,0.47,0.30}{#1}}
\newcommand{\WarningTok}[1]{\textcolor[rgb]{0.37,0.37,0.37}{\textit{#1}}}

\providecommand{\tightlist}{%
  \setlength{\itemsep}{0pt}\setlength{\parskip}{0pt}}\usepackage{longtable,booktabs,array}
\usepackage{calc} % for calculating minipage widths
% Correct order of tables after \paragraph or \subparagraph
\usepackage{etoolbox}
\makeatletter
\patchcmd\longtable{\par}{\if@noskipsec\mbox{}\fi\par}{}{}
\makeatother
% Allow footnotes in longtable head/foot
\IfFileExists{footnotehyper.sty}{\usepackage{footnotehyper}}{\usepackage{footnote}}
\makesavenoteenv{longtable}
\usepackage{graphicx}
\makeatletter
\def\maxwidth{\ifdim\Gin@nat@width>\linewidth\linewidth\else\Gin@nat@width\fi}
\def\maxheight{\ifdim\Gin@nat@height>\textheight\textheight\else\Gin@nat@height\fi}
\makeatother
% Scale images if necessary, so that they will not overflow the page
% margins by default, and it is still possible to overwrite the defaults
% using explicit options in \includegraphics[width, height, ...]{}
\setkeys{Gin}{width=\maxwidth,height=\maxheight,keepaspectratio}
% Set default figure placement to htbp
\makeatletter
\def\fps@figure{htbp}
\makeatother

\KOMAoption{captions}{tableheading}
\makeatletter
\@ifpackageloaded{tcolorbox}{}{\usepackage[many]{tcolorbox}}
\@ifpackageloaded{fontawesome5}{}{\usepackage{fontawesome5}}
\definecolor{quarto-callout-color}{HTML}{909090}
\definecolor{quarto-callout-note-color}{HTML}{0758E5}
\definecolor{quarto-callout-important-color}{HTML}{CC1914}
\definecolor{quarto-callout-warning-color}{HTML}{EB9113}
\definecolor{quarto-callout-tip-color}{HTML}{00A047}
\definecolor{quarto-callout-caution-color}{HTML}{FC5300}
\definecolor{quarto-callout-color-frame}{HTML}{acacac}
\definecolor{quarto-callout-note-color-frame}{HTML}{4582ec}
\definecolor{quarto-callout-important-color-frame}{HTML}{d9534f}
\definecolor{quarto-callout-warning-color-frame}{HTML}{f0ad4e}
\definecolor{quarto-callout-tip-color-frame}{HTML}{02b875}
\definecolor{quarto-callout-caution-color-frame}{HTML}{fd7e14}
\makeatother
\makeatletter
\makeatother
\makeatletter
\@ifpackageloaded{bookmark}{}{\usepackage{bookmark}}
\makeatother
\makeatletter
\@ifpackageloaded{caption}{}{\usepackage{caption}}
\AtBeginDocument{%
\ifdefined\contentsname
  \renewcommand*\contentsname{Table of contents}
\else
  \newcommand\contentsname{Table of contents}
\fi
\ifdefined\listfigurename
  \renewcommand*\listfigurename{List of Figures}
\else
  \newcommand\listfigurename{List of Figures}
\fi
\ifdefined\listtablename
  \renewcommand*\listtablename{List of Tables}
\else
  \newcommand\listtablename{List of Tables}
\fi
\ifdefined\figurename
  \renewcommand*\figurename{Figure}
\else
  \newcommand\figurename{Figure}
\fi
\ifdefined\tablename
  \renewcommand*\tablename{Table}
\else
  \newcommand\tablename{Table}
\fi
}
\@ifpackageloaded{float}{}{\usepackage{float}}
\floatstyle{ruled}
\@ifundefined{c@chapter}{\newfloat{codelisting}{h}{lop}}{\newfloat{codelisting}{h}{lop}[chapter]}
\floatname{codelisting}{Listing}
\newcommand*\listoflistings{\listof{codelisting}{List of Listings}}
\makeatother
\makeatletter
\@ifpackageloaded{caption}{}{\usepackage{caption}}
\@ifpackageloaded{subcaption}{}{\usepackage{subcaption}}
\makeatother
\makeatletter
\@ifpackageloaded{tcolorbox}{}{\usepackage[many]{tcolorbox}}
\makeatother
\makeatletter
\@ifundefined{shadecolor}{\definecolor{shadecolor}{rgb}{.97, .97, .97}}
\makeatother
\makeatletter
\makeatother
\ifLuaTeX
  \usepackage{selnolig}  % disable illegal ligatures
\fi
\IfFileExists{bookmark.sty}{\usepackage{bookmark}}{\usepackage{hyperref}}
\IfFileExists{xurl.sty}{\usepackage{xurl}}{} % add URL line breaks if available
\urlstyle{same} % disable monospaced font for URLs
\hypersetup{
  pdftitle={ROS 2 : Mea Via et Mea Disciplina},
  pdfauthor={Rishikesavan Ramesh, Manoj M},
  colorlinks=true,
  linkcolor={blue},
  filecolor={Maroon},
  citecolor={Blue},
  urlcolor={Blue},
  pdfcreator={LaTeX via pandoc}}

\title{ROS 2 : Mea Via et Mea Disciplina}
\author{Rishikesavan Ramesh, Manoj M}
\date{4/5/23}

\begin{document}
\maketitle
\ifdefined\Shaded\renewenvironment{Shaded}{\begin{tcolorbox}[enhanced, breakable, boxrule=0pt, frame hidden, interior hidden, sharp corners, borderline west={3pt}{0pt}{shadecolor}]}{\end{tcolorbox}}\fi

\renewcommand*\contentsname{Table of contents}
{
\hypersetup{linkcolor=}
\setcounter{tocdepth}{2}
\tableofcontents
}
\bookmarksetup{startatroot}

\hypertarget{preface}{%
\chapter*{Preface}\label{preface}}
\addcontentsline{toc}{chapter}{Preface}

\markboth{Preface}{Preface}

Welcome to our journey with ROS2! As authors of this e-book, we are
excited to document our experience as beginner learners in the field of
robotics and automation. Our goal is to share our learning journey with
you, in the hopes that it will be useful for others who are new to ROS2
as well.

We are two friends, Rishikesavan Ramesh and Manoj M, who share a passion
for robotics and automation. Together, we decided to explore ROS2 as a
way to build our skills and gain practical experience in the field.
Along the way, we encountered many challenges, took risks, and learned
valuable lessons that we want to share with others.

In this e-book, we will take you through our journey with ROS2, from the
basics of getting started to more advanced topics like application
development and contributing to the ROS2 community. We have organized
the content in a way that we hope will be easy to follow, with plenty of
examples and practical tips to help you along the way.

Our approach to learning involves documenting our experiences and
creating a resource that can help other beginners on the same path. We
believe that this approach can be very useful, as it provides a
real-world perspective on the learning process and can help new
beginners avoid some of the common pitfalls and mistakes.

Whether you are a beginner just starting out with ROS2 or an
intermediate learner looking to expand your skills, we hope that this
e-book will be a valuable resource for you. So join us on our
adventurous path with ROS2, and let's explore the exciting world of
robotics and automation together!

You can find our LinkedIn profiles at

\url{https://www.linkedin.com/in/manoj-murali-/} and
\url{https://www.linkedin.com/in/rishikesavan-ramesh/}.

\bookmarksetup{startatroot}

\hypertarget{introduction-to-ros2}{%
\chapter{Introduction to ROS2}\label{introduction-to-ros2}}

\begin{tcolorbox}[enhanced jigsaw, coltitle=black, breakable, title=\textcolor{quarto-callout-important-color}{\faExclamation}\hspace{0.5em}{Important}, toprule=.15mm, leftrule=.75mm, colframe=quarto-callout-important-color-frame, colbacktitle=quarto-callout-important-color!10!white, opacitybacktitle=0.6, left=2mm, colback=white, toptitle=1mm, bottomtitle=1mm, titlerule=0mm, arc=.35mm, rightrule=.15mm, opacityback=0, bottomrule=.15mm]

You are reading the work-in-progress edition of ROS 2 : Mea Via et Mea
Disciplina. This chapter is currently a dumping ground for ideas, and we
don't recommend reading it. The complete version will be available here
soon. Stay tuned!

\end{tcolorbox}

ROS (Robot Operating System) was first released by Open Robotics in 2007
as a set of software libraries and tools for building robot
applications. At the core of a ROS system are independent nodes that
communicate with each other using a publish/subscribe messaging
platform. For example, a sensor's driver could be implemented as a node
that publishes sensor data, and this data can be read by any number of
other nodes, such as filters, loggers, mapping, and navigation nodes.
What's great about ROS is that these nodes can be on different systems
or used by different architectures, making it flexible and adaptable to
users' needs.

ROS 2 is the next version of ROS and includes many of the same
components and tools as its predecessor, but also adds new features like
an improved communication stack with real-time data distribution service
(DDS) protocol, support for multiple DDS implementations, and DDS
security support. It also offers improved logging capabilities, ability
to configure QoS at startup, improved rosbag2 performance, and more. ROS
2 is designed to be scalable, efficient, and secure, making it a great
choice for building robotic applications.

\bookmarksetup{startatroot}

\hypertarget{opensidebar-false}{%
\chapter{openSidebar: false}\label{opensidebar-false}}

\begin{tcolorbox}[enhanced jigsaw, coltitle=black, breakable, title=\textcolor{quarto-callout-important-color}{\faExclamation}\hspace{0.5em}{Important}, toprule=.15mm, leftrule=.75mm, colframe=quarto-callout-important-color-frame, colbacktitle=quarto-callout-important-color!10!white, opacitybacktitle=0.6, left=2mm, colback=white, toptitle=1mm, bottomtitle=1mm, titlerule=0mm, arc=.35mm, rightrule=.15mm, opacityback=0, bottomrule=.15mm]

You are reading the work-in-progress edition of ROS 2 : Mea Via et Mea
Disciplina. This chapter is currently a dumping ground for ideas, and we
don't recommend reading it. The complete version will be available here
soon. Stay tuned!

\end{tcolorbox}

\begin{center}\rule{0.5\linewidth}{0.5pt}\end{center}

website: comments: hypothesis: theme: clean

\bookmarksetup{startatroot}

\hypertarget{installation}{%
\chapter{Installation}\label{installation}}

We recommend and use ROS2: humble (for now).

\bookmarksetup{startatroot}

\hypertarget{sec-ubuntu}{%
\chapter{Ubuntu}\label{sec-ubuntu}}

\hypertarget{set-locale}{%
\subsubsection{Set Locale}\label{set-locale}}

\begin{Shaded}
\begin{Highlighting}[]

\ExtensionTok{locale}  \CommentTok{\# check for UTF{-}8}

\FunctionTok{sudo}\NormalTok{ apt update }\KeywordTok{\&\&} \FunctionTok{sudo}\NormalTok{ apt install locales}
\FunctionTok{sudo}\NormalTok{ locale{-}gen en\_US en\_US.UTF{-}8}
\FunctionTok{sudo}\NormalTok{ update{-}locale LC\_ALL=en\_US.UTF{-}8 LANG=en\_US.UTF{-}8}
\BuiltInTok{export} \VariableTok{LANG}\OperatorTok{=}\NormalTok{en\_US.UTF{-}8}

\ExtensionTok{locale}  \CommentTok{\# verify settings}
\end{Highlighting}
\end{Shaded}

\hypertarget{setup-sources}{%
\subsubsection{Setup Sources}\label{setup-sources}}

\begin{Shaded}
\begin{Highlighting}[]
\FunctionTok{sudo}\NormalTok{ apt install software{-}properties{-}common}
\FunctionTok{sudo}\NormalTok{ add{-}apt{-}repository universe}
\FunctionTok{sudo}\NormalTok{ apt update }\KeywordTok{\&\&} \FunctionTok{sudo}\NormalTok{ apt install curl }\AttributeTok{{-}y}
\FunctionTok{sudo}\NormalTok{ curl }\AttributeTok{{-}sSL}\NormalTok{ https://raw.githubusercontent.com/ros/rosdistro/master/ros.key }\AttributeTok{{-}o}\NormalTok{ /usr/share/keyrings/ros{-}archive{-}keyring.gpg}
\BuiltInTok{echo} \StringTok{"deb [arch=}\VariableTok{$(}\ExtensionTok{dpkg} \AttributeTok{{-}{-}print{-}architecture}\VariableTok{)}\StringTok{ signed{-}by=/usr/share/keyrings/ros{-}archive{-}keyring.gpg] http://packages.ros.org/ros2/ubuntu }\VariableTok{$(}\BuiltInTok{.}\NormalTok{ /etc/os{-}release }\KeywordTok{\&\&} \BuiltInTok{echo} \VariableTok{$UBUNTU\_CODENAME)}\StringTok{ main"} \KeywordTok{|} \FunctionTok{sudo}\NormalTok{ tee /etc/apt/sources.list.d/ros2.list }\OperatorTok{\textgreater{}}\NormalTok{ /dev/null}
\end{Highlighting}
\end{Shaded}

\hypertarget{install-ros2-packages}{%
\subsubsection{Install ROS2 packages}\label{install-ros2-packages}}

\begin{Shaded}
\begin{Highlighting}[]
\FunctionTok{sudo}\NormalTok{ apt update}
\FunctionTok{sudo}\NormalTok{ apt upgrade}
\end{Highlighting}
\end{Shaded}

here you have three options,

\begin{Shaded}
\begin{Highlighting}[]
\FunctionTok{sudo}\NormalTok{ apt install ros{-}humble{-}desktop}
\end{Highlighting}
\end{Shaded}

or

\begin{Shaded}
\begin{Highlighting}[]
\FunctionTok{sudo}\NormalTok{ apt install ros{-}humble{-}ros{-}base}
\end{Highlighting}
\end{Shaded}

or

\begin{Shaded}
\begin{Highlighting}[]
\FunctionTok{sudo}\NormalTok{ apt install ros{-}humble{-}ros{-}dev{-}tools}
\end{Highlighting}
\end{Shaded}

\hypertarget{environment-setup}{%
\subsubsection{Environment setup}\label{environment-setup}}

\begin{Shaded}
\begin{Highlighting}[]
\CommentTok{\# Replace ".bash" with your shell if you\textquotesingle{}re not using bash}
\CommentTok{\# Possible values are: setup.bash, setup.sh, setup.zsh}
\BuiltInTok{source}\NormalTok{ /opt/ros/humble/setup.bash}
\end{Highlighting}
\end{Shaded}

\hypertarget{uninstall}{%
\subsubsection{Uninstall}\label{uninstall}}

\begin{Shaded}
\begin{Highlighting}[]
\FunctionTok{sudo}\NormalTok{ apt remove \textasciitilde{}nros{-}humble{-}}\PreprocessorTok{*} \KeywordTok{\&\&} \FunctionTok{sudo}\NormalTok{ apt autoremove}
\FunctionTok{sudo}\NormalTok{ rm /etc/apt/sources.list.d/ros2.list}
\FunctionTok{sudo}\NormalTok{ apt update}
\FunctionTok{sudo}\NormalTok{ apt autoremove}
\CommentTok{\# Consider upgrading for packages previously shadowed.}
\FunctionTok{sudo}\NormalTok{ apt upgrade}
\end{Highlighting}
\end{Shaded}

\bookmarksetup{startatroot}

\hypertarget{sec-rhel-8}{%
\chapter{RHEL 8}\label{sec-rhel-8}}

\hypertarget{set-locale-1}{%
\subsubsection{Set Locale}\label{set-locale-1}}

\begin{Shaded}
\begin{Highlighting}[]
\ExtensionTok{locale}  \CommentTok{\# check for UTF{-}8}

\FunctionTok{sudo}\NormalTok{ dnf install langpacks{-}en glibc{-}langpack{-}en}
\BuiltInTok{export} \VariableTok{LANG}\OperatorTok{=}\NormalTok{en\_US.UTF{-}8}

\ExtensionTok{locale}  \CommentTok{\# verify settings}
\end{Highlighting}
\end{Shaded}

\hypertarget{setup-sources-1}{%
\subsubsection{Setup Sources}\label{setup-sources-1}}

\begin{Shaded}
\begin{Highlighting}[]
\FunctionTok{sudo}\NormalTok{ dnf install }\StringTok{\textquotesingle{}dnf{-}command(config{-}manager)\textquotesingle{}}\NormalTok{ epel{-}release }\AttributeTok{{-}y}
\FunctionTok{sudo}\NormalTok{ dnf config{-}manager }\AttributeTok{{-}{-}set{-}enabled}\NormalTok{ powertools}
\end{Highlighting}
\end{Shaded}

\begin{tcolorbox}[enhanced jigsaw, coltitle=black, breakable, title=\textcolor{quarto-callout-note-color}{\faInfo}\hspace{0.5em}{Note}, toprule=.15mm, leftrule=.75mm, colframe=quarto-callout-note-color-frame, colbacktitle=quarto-callout-note-color!10!white, opacitybacktitle=0.6, left=2mm, colback=white, toptitle=1mm, bottomtitle=1mm, titlerule=0mm, arc=.35mm, rightrule=.15mm, opacityback=0, bottomrule=.15mm]

This step may be slightly different depending on the distribution you
are using. Check the EPEL
documentation:\url{https://docs.fedoraproject.org/en-US/epel/\#_quickstart}

\end{tcolorbox}

\begin{Shaded}
\begin{Highlighting}[]
\FunctionTok{sudo}\NormalTok{ dnf install curl}
\FunctionTok{sudo}\NormalTok{ curl }\AttributeTok{{-}{-}output}\NormalTok{ /etc/yum.repos.d/ros2.repo http://packages.ros.org/ros2/rhel/ros2.repo}
\FunctionTok{sudo}\NormalTok{ dnf makecache}
\end{Highlighting}
\end{Shaded}

\hypertarget{install-ros2-packages-1}{%
\subsubsection{Install ROS2 packages}\label{install-ros2-packages-1}}

\begin{Shaded}
\begin{Highlighting}[]
\FunctionTok{sudo}\NormalTok{ dnf update}
\FunctionTok{sudo}\NormalTok{ dnf install ros{-}humble{-}desktop}
\end{Highlighting}
\end{Shaded}

or simply

\begin{Shaded}
\begin{Highlighting}[]
\FunctionTok{sudo}\NormalTok{ dnf install ros{-}humble{-}ros{-}base}
\end{Highlighting}
\end{Shaded}

\hypertarget{environment-setup-1}{%
\subsubsection{Environment setup}\label{environment-setup-1}}

\begin{Shaded}
\begin{Highlighting}[]
\CommentTok{\# Replace ".bash" with your shell if you\textquotesingle{}re not using bash}
\CommentTok{\# Possible values are: setup.bash, setup.sh, setup.zsh}
\BuiltInTok{source}\NormalTok{ /opt/ros/humble/setup.bash}
\end{Highlighting}
\end{Shaded}

\hypertarget{uninstall-1}{%
\subsubsection{Uninstall}\label{uninstall-1}}

\begin{Shaded}
\begin{Highlighting}[]
\FunctionTok{sudo}\NormalTok{ dnf remove ros{-}humble{-}}\PreprocessorTok{*}
\end{Highlighting}
\end{Shaded}

\bookmarksetup{startatroot}

\hypertarget{windows}{%
\chapter{Windows}\label{windows}}

\begin{tcolorbox}[enhanced jigsaw, coltitle=black, breakable, title=\textcolor{quarto-callout-caution-color}{\faFire}\hspace{0.5em}{Danger}, toprule=.15mm, leftrule=.75mm, colframe=quarto-callout-caution-color-frame, colbacktitle=quarto-callout-caution-color!10!white, opacitybacktitle=0.6, left=2mm, colback=white, toptitle=1mm, bottomtitle=1mm, titlerule=0mm, arc=.35mm, rightrule=.15mm, opacityback=0, bottomrule=.15mm]

Sorry, we dont do that here, although it is quite possible with ros2

\end{tcolorbox}

\bookmarksetup{startatroot}

\hypertarget{sourcing}{%
\chapter{Sourcing}\label{sourcing}}

\begin{tcolorbox}[enhanced jigsaw, coltitle=black, breakable, title=\textcolor{quarto-callout-important-color}{\faExclamation}\hspace{0.5em}{Important}, toprule=.15mm, leftrule=.75mm, colframe=quarto-callout-important-color-frame, colbacktitle=quarto-callout-important-color!10!white, opacitybacktitle=0.6, left=2mm, colback=white, toptitle=1mm, bottomtitle=1mm, titlerule=0mm, arc=.35mm, rightrule=.15mm, opacityback=0, bottomrule=.15mm]

You are reading the work-in-progress edition of ROS 2 : Mea Via et Mea
Disciplina. This chapter is currently a dumping ground for ideas, and we
don't recommend reading it. The complete version will be available here
soon. Stay tuned!

\end{tcolorbox}

\hypertarget{sourcing-the-binaries}{%
\section{Sourcing the binaries}\label{sourcing-the-binaries}}

By default, when you install ROS2, it does not source its binaries
automatically. This means that you will need to manually source the
setup file to use the ROS2 binaries.

There are several reasons why ROS2 doesn't source its binaries by
default:

\hypertarget{isolation}{%
\subsection{Isolation:}\label{isolation}}

ROS2 supports multiple installations of itself and its dependencies, so
not sourcing the binaries by default allows for better isolation between
different installations. This means that you can have multiple ROS2
installations on the same machine without them interfering with each
other.

\hypertarget{flexibility}{%
\subsection{Flexibility:}\label{flexibility}}

By not sourcing the binaries by default, ROS2 allows you to choose which
version of the software to use for each terminal session. This is useful
when working on different projects that require different versions of
ROS2.

\hypertarget{avoiding-conflicts}{%
\subsection{Avoiding conflicts:}\label{avoiding-conflicts}}

Sourcing the binaries by default can cause conflicts with other software
installed on your machine that may use the same environment variables.

In summary, not sourcing the binaries by default in ROS2 allows for
better isolation, flexibility, and avoids conflicts with other software
on your machine. However, it does require a manual step to set up each
terminal session to use ROS2, which can be seen as a minor
inconvenience.

\begin{tcolorbox}[enhanced jigsaw, coltitle=black, breakable, title=\textcolor{quarto-callout-tip-color}{\faLightbulb}\hspace{0.5em}{Tip}, toprule=.15mm, leftrule=.75mm, colframe=quarto-callout-tip-color-frame, colbacktitle=quarto-callout-tip-color!10!white, opacitybacktitle=0.6, left=2mm, colback=white, toptitle=1mm, bottomtitle=1mm, titlerule=0mm, arc=.35mm, rightrule=.15mm, opacityback=0, bottomrule=.15mm]

This can be advantageous when you want to work on multiple projects that
require you to install multiple distros.

Creating a shortcut would be helpful. Here is how,

\textbf{Step 1}: Create a file in your home directory,

\begin{Shaded}
\begin{Highlighting}[]
\FunctionTok{touch}\NormalTok{ \textasciitilde{}/.ros2\_config}
\end{Highlighting}
\end{Shaded}

\textbf{Step 2}: Edit \textasciitilde/.bashrc and
\textasciitilde/.ros2\_config

\begin{Shaded}
\begin{Highlighting}[]
\BuiltInTok{echo} \StringTok{"alias shumble="}\NormalTok{source \textasciitilde{}/.ros2\_config}\StringTok{""} \OperatorTok{\textgreater{}\textgreater{}}\NormalTok{ \textasciitilde{}/.bashrc}
\BuiltInTok{echo} \StringTok{"source /opt/ros/humble/setup.bash"} \OperatorTok{\textgreater{}}\NormalTok{ \textasciitilde{}/.ros2\_config}
\end{Highlighting}
\end{Shaded}

It is advisable to edit the \textasciitilde/.bashrc as less as possible,
so that reducing the risk of misconfiguring the vital configurations in
the \textasciitilde/.bashrc

You can now twerk \textasciitilde/.ros2\_config to change any ros
related settings, while it being isolated from other configurations.

\end{tcolorbox}

\bookmarksetup{startatroot}

\hypertarget{command-line-tools}{%
\chapter{Command line tools}\label{command-line-tools}}

\begin{tcolorbox}[enhanced jigsaw, coltitle=black, breakable, title=\textcolor{quarto-callout-important-color}{\faExclamation}\hspace{0.5em}{Important}, toprule=.15mm, leftrule=.75mm, colframe=quarto-callout-important-color-frame, colbacktitle=quarto-callout-important-color!10!white, opacitybacktitle=0.6, left=2mm, colback=white, toptitle=1mm, bottomtitle=1mm, titlerule=0mm, arc=.35mm, rightrule=.15mm, opacityback=0, bottomrule=.15mm]

You are reading the work-in-progress edition of ROS 2 : Mea Via et Mea
Disciplina. This chapter is currently a dumping ground for ideas, and we
don't recommend reading it. The complete version will be available here
soon. Stay tuned!

\end{tcolorbox}

\begin{longtable}[]{@{}
  >{\raggedright\arraybackslash}p{(\columnwidth - 2\tabcolsep) * \real{0.3448}}
  >{\raggedright\arraybackslash}p{(\columnwidth - 2\tabcolsep) * \real{0.6552}}@{}}
\toprule()
\begin{minipage}[b]{\linewidth}\raggedright
Commands
\end{minipage} & \begin{minipage}[b]{\linewidth}\raggedright
Short description
\end{minipage} \\
\midrule()
\endhead
ros2 run & Run a ROS2 node from a package. \\
ros2 node & Interact with a running ROS2 node. \\
ros2 topic & Interact with ROS2 topics. \\
ros2 service & Interact with ROS2 services. \\
ros2 param & Interact with ROS2 parameters. \\
ros2 bag & Record and play back ROS2 messages. \\
ros2 launch & Launch multiple nodes at once. \\
ros2 interface & Interact with ROS2 interfaces. \\
ros2 pkg & Manage ROS2 packages. \\
ros2 pkg create & Create a new ROS2 package. \\
ros2 pkg list & List available ROS2 packages. \\
ros2 pkg create --build-type ament\_cmake & Create a new ROS2 package
with the Ament CMake build system. \\
ros2 pkg create --build-type ament\_python & Create a new ROS2 package
with the Ament Python build system. \\
\bottomrule()
\end{longtable}

\begin{tcolorbox}[enhanced jigsaw, coltitle=black, breakable, title=\textcolor{quarto-callout-note-color}{\faInfo}\hspace{0.5em}{Expand To Learn More About \textbf{``ros2 pkg create''}}, toprule=.15mm, leftrule=.75mm, colframe=quarto-callout-note-color-frame, colbacktitle=quarto-callout-note-color!10!white, opacitybacktitle=0.6, left=2mm, colback=white, toptitle=1mm, bottomtitle=1mm, titlerule=0mm, arc=.35mm, rightrule=.15mm, opacityback=0, bottomrule=.15mm]

\begin{longtable}[]{@{}
  >{\raggedright\arraybackslash}p{(\columnwidth - 2\tabcolsep) * \real{0.3448}}
  >{\raggedright\arraybackslash}p{(\columnwidth - 2\tabcolsep) * \real{0.6552}}@{}}
\toprule()
\begin{minipage}[b]{\linewidth}\raggedright
Commands
\end{minipage} & \begin{minipage}[b]{\linewidth}\raggedright
Short description
\end{minipage} \\
\midrule()
\endhead
ros2 pkg create --build-type ament\_lint & Create a new ROS2 package
with the Ament Lint build system. \\
ros2 pkg create --build-type ament\_cmake\_pytest & Create a new ROS2
package with the Ament CMake Pytest build system. \\
ros2 pkg create --build-type ament\_cmake\_gtest & Create a new ROS2
package with the Ament CMake Gtest build system. \\
ros2 pkg create --build-type colcon & Create a new ROS2 package with the
Colcon build system. \\
ros2 pkg create --build-type colcon-python & Create a new ROS2 package
with the Colcon Python build system. \\
ros2 pkg create --build-type colcon-cmake & Create a new ROS2 package
with the Colcon CMake build system. \\
ros2 pkg create --build-type colcon-ros & Create a new ROS2 package with
the Colcon ROS build system. \\
ros2 pkg create --build-type colcon-ros-python & Create a new ROS2
package with the Colcon ROS Python build system. \\
ros2 pkg create --build-type colcon-ros-cmake & Create a new ROS2
package with the Colcon ROS CMake build system. \\
ros2 pkg create --build-type ament\_cmake\_ros & Create a new ROS2
package with the Ament CMake ROS build system. \\
ros2 pkg create --build-type ament\_cmake\_python & Create a new ROS2
package with the Ament CMake Python build system. \\
ros2 pkg create --build-type ament\_cmake\_lint & Create a new ROS2
package with the Ament CMake Lint build system. \\
ros2 pkg create --build-type ament\_cmake\_pytest\_coverage & Create a
new ROS2 package with the Ament CMake Pytest Coverage build system. \\
ros2 pkg create --build-type ament\_cmake\_ros\_testing & Create a new
ROS2 package with the Ament CMake ROS Testing build system. \\
ros2 pkg create --build-type ament\_cmake\_flake8 & Create a new ROS2
package with the Ament CMake Flake8 build system. \\
ros2 pkg create --build-type ament\_cmake\_mypy & Create a new ROS2
package with the Ament CMake MyPy build system. \\
ros2 pkg create --build-type ament\_cmake\_pep257 & Create a new ROS2
package with \\
\bottomrule()
\end{longtable}

\end{tcolorbox}

\bookmarksetup{startatroot}

\hypertarget{nodes}{%
\chapter{Nodes}\label{nodes}}

\begin{tcolorbox}[enhanced jigsaw, coltitle=black, breakable, title=\textcolor{quarto-callout-important-color}{\faExclamation}\hspace{0.5em}{Important}, toprule=.15mm, leftrule=.75mm, colframe=quarto-callout-important-color-frame, colbacktitle=quarto-callout-important-color!10!white, opacitybacktitle=0.6, left=2mm, colback=white, toptitle=1mm, bottomtitle=1mm, titlerule=0mm, arc=.35mm, rightrule=.15mm, opacityback=0, bottomrule=.15mm]

You are reading the work-in-progress edition of ROS 2 : Mea Via et Mea
Disciplina. This chapter is currently a dumping ground for ideas, and we
don't recommend reading it. The complete version will be available here
soon. Stay tuned!

\end{tcolorbox}

In ROS2, a node is the fundamental unit of computation that performs a
specific task in a robotic system. It is a process that communicates
with other nodes in the system through topics, services, and parameters.

\begin{figure}

{\centering \includegraphics{./images/nodes.gif}

}

\caption{Source
\href{https://docs.ros.org/en/foxy/Tutorials/Beginner-CLI-Tools/Understanding-ROS2-Nodes/Understanding-ROS2-Nodes.html}{ros.org}}

\end{figure}

Nodes are often designed to perform a single function or task, such as
reading data from a sensor, processing that data, or controlling an
actuator. Nodes can be written in a variety of programming languages,
including C++, Python, and others.

Nodes communicate with each other by publishing and subscribing to
topics. A topic is a named bus over which nodes can send and receive
messages. When a node publishes a message to a topic, all other nodes
that are subscribed to that topic will receive the message.

Nodes can also provide and consume services. A service is a named
request/response pair of messages that a node can use to get or set
information or perform a specific action. A node that provides a service
listens for requests, performs the requested action, and sends back a
response.

Nodes can also use parameters to store and retrieve configuration data.
A parameter is a named value that can be set and read by nodes in the
system. Parameters can be used to configure a node's behavior or to
store information such as calibration data.

In summary, nodes are the building blocks of a ROS2 system, and they
communicate with each other through topics, services, and parameters.
Each node performs a specific function or task, and together they form a
distributed system that can be used to control robots or other complex
systems.

\bookmarksetup{startatroot}

\hypertarget{topic}{%
\chapter{Topic}\label{topic}}

\begin{tcolorbox}[enhanced jigsaw, coltitle=black, breakable, title=\textcolor{quarto-callout-important-color}{\faExclamation}\hspace{0.5em}{Important}, toprule=.15mm, leftrule=.75mm, colframe=quarto-callout-important-color-frame, colbacktitle=quarto-callout-important-color!10!white, opacitybacktitle=0.6, left=2mm, colback=white, toptitle=1mm, bottomtitle=1mm, titlerule=0mm, arc=.35mm, rightrule=.15mm, opacityback=0, bottomrule=.15mm]

You are reading the work-in-progress edition of ROS 2 : Mea Via et Mea
Disciplina. This chapter is currently a dumping ground for ideas, and we
don't recommend reading it. The complete version will be available here
soon. Stay tuned!

\end{tcolorbox}

In ROS2, a topic is a named bus over which nodes can send and receive
messages. Topics are used for asynchronous communication between nodes,
allowing them to exchange data without needing to know anything about
each other's internal workings.

Each topic has a unique name, which is used by nodes to identify the
topic they want to communicate on. Nodes can publish messages to a topic
or subscribe to a topic to receive messages. When a node publishes a
message to a topic, all other nodes that are subscribed to that topic
will receive the message. \includegraphics{./images/topic.gif}1 Messages
sent over a topic can be of any type, as long as they are defined in the
same message package and have the same message name. For example, a
sensor node might publish messages containing sensor readings, while a
control node might subscribe to those messages and use the data to
control a robot's movements.

ROS2 uses a publisher-subscriber model for topic communication.
Publishers are nodes that send messages to a topic, while subscribers
are nodes that receive messages from a topic. Nodes can publish and
subscribe to multiple topics at the same time, allowing for complex
communication patterns between nodes.

ROS2 topics are an important part of building a distributed robotic
system, as they allow nodes to communicate in a decoupled manner. By
using topics, nodes can exchange data without needing to know anything
about each other's internal workings, making it easier to build complex
robotic systems out of smaller, specialized components.

\bookmarksetup{startatroot}

\hypertarget{messages}{%
\chapter{Messages}\label{messages}}

\begin{tcolorbox}[enhanced jigsaw, coltitle=black, breakable, title=\textcolor{quarto-callout-important-color}{\faExclamation}\hspace{0.5em}{Important}, toprule=.15mm, leftrule=.75mm, colframe=quarto-callout-important-color-frame, colbacktitle=quarto-callout-important-color!10!white, opacitybacktitle=0.6, left=2mm, colback=white, toptitle=1mm, bottomtitle=1mm, titlerule=0mm, arc=.35mm, rightrule=.15mm, opacityback=0, bottomrule=.15mm]

You are reading the work-in-progress edition of ROS 2 : Mea Via et Mea
Disciplina. This chapter is currently a dumping ground for ideas, and we
don't recommend reading it. The complete version will be available here
soon. Stay tuned!

\end{tcolorbox}

In ROS2, messages are the basic units of data that are exchanged between
nodes over topics. A message is a data structure that is defined in a
message package and contains information about a specific topic.

Messages can be of any type, as long as they are defined in the same
message package and have the same message name. For example, a message
might contain information about a robot's position and orientation, or
about a sensor reading such as temperature or distance.

ROS2 messages are typically defined using the ROS2 interface description
language (IDL), which is a language-agnostic way of describing data
structures. This makes it possible to define messages in a variety of
programming languages, including C++, Python, and others.

When a node publishes a message to a topic, it sends the message data
over the network to all nodes that are subscribed to that topic. Each
subscribed node can then receive the message and process the data as
needed.

Messages are a key component of building a distributed robotic system,
as they allow nodes to communicate with each other in a decoupled
manner. By defining messages that describe specific pieces of data,
nodes can exchange information without needing to know anything about
each other's internal workings. This makes it easier to build complex
robotic systems out of smaller, specialized components.

In summary, messages in ROS2 are the basic units of data that are
exchanged between nodes over topics. They are defined using a
language-agnostic data structure description language and can be of any
type. Messages allow nodes to communicate in a decoupled manner, which
is essential for building complex robotic systems out of smaller,
specialized components.

\bookmarksetup{startatroot}

\hypertarget{services}{%
\chapter{Services}\label{services}}

\begin{tcolorbox}[enhanced jigsaw, coltitle=black, breakable, title=\textcolor{quarto-callout-important-color}{\faExclamation}\hspace{0.5em}{Important}, toprule=.15mm, leftrule=.75mm, colframe=quarto-callout-important-color-frame, colbacktitle=quarto-callout-important-color!10!white, opacitybacktitle=0.6, left=2mm, colback=white, toptitle=1mm, bottomtitle=1mm, titlerule=0mm, arc=.35mm, rightrule=.15mm, opacityback=0, bottomrule=.15mm]

You are reading the work-in-progress edition of ROS 2 : Mea Via et Mea
Disciplina. This chapter is currently a dumping ground for ideas, and we
don't recommend reading it. The complete version will be available here
soon. Stay tuned!

\end{tcolorbox}

ROS2, services are a way for nodes to request a specific operation from
another node and receive a response. Services are similar to topics, but
instead of sending and receiving messages, nodes send requests and
receive responses.

Services are defined using a service description file that specifies the
name of the service, the types of the request and response messages, and
any necessary parameters or options. The service definition is used to
generate code that can be used by nodes to implement or use the service.

When a node wants to use a service, it sends a request message to the
node that provides the service. The service provider receives the
request message, performs the requested operation, and sends a response
message back to the requester. Once the response is received, the
requester can continue with its operation.

Services are typically used for operations that require a direct
response, such as configuring a sensor, performing a calculation, or
sending a command to a robot. Unlike topics, services are synchronous,
meaning that the requester will block until it receives a response from
the service provider.

In ROS2, services are an important part of building distributed robotic
systems, as they allow nodes to request specific operations from other
nodes and receive a response. By using services, nodes can communicate
in a decoupled manner and perform complex operations without needing to
know anything about each other's internal workings.

In summary, services in ROS2 are a way for nodes to request a specific
operation from another node and receive a response. Services are defined
using a service description file and are used for operations that
require a direct response. Services are an important part of building
distributed robotic systems, as they allow nodes to communicate in a
decoupled manner and perform complex operations without needing to know
anything about each other's internal workings.

\bookmarksetup{startatroot}

\hypertarget{ros2-concepts}{%
\chapter{Ros2 Concepts}\label{ros2-concepts}}

\begin{tcolorbox}[enhanced jigsaw, coltitle=black, breakable, title=\textcolor{quarto-callout-important-color}{\faExclamation}\hspace{0.5em}{Important}, toprule=.15mm, leftrule=.75mm, colframe=quarto-callout-important-color-frame, colbacktitle=quarto-callout-important-color!10!white, opacitybacktitle=0.6, left=2mm, colback=white, toptitle=1mm, bottomtitle=1mm, titlerule=0mm, arc=.35mm, rightrule=.15mm, opacityback=0, bottomrule=.15mm]

You are reading the work-in-progress edition of ROS 2 : Mea Via et Mea
Disciplina. This chapter is currently a dumping ground for ideas, and we
don't recommend reading it. The complete version will be available here
soon. Stay tuned!

\end{tcolorbox}

\begin{longtable}[]{@{}
  >{\raggedright\arraybackslash}p{(\columnwidth - 2\tabcolsep) * \real{0.3939}}
  >{\raggedright\arraybackslash}p{(\columnwidth - 2\tabcolsep) * \real{0.6061}}@{}}
\toprule()
\begin{minipage}[b]{\linewidth}\raggedright
ROS2-Concept
\end{minipage} & \begin{minipage}[b]{\linewidth}\raggedright
Description
\end{minipage} \\
\midrule()
\endhead
Nodes & A node is a process that performs a specific task, such as
sensing or actuating. Nodes can communicate with each other by sending
and receiving messages over topics. \\
Topics & A topic is a named bus over which nodes can publish and
subscribe to messages. Topics are used for one-to-many communication. \\
Services & A service is a request-response communication model between
nodes. A node can make a request to a service and receive a response
from it. \\
Actions & Actions are similar to services, but they allow for more
complex communication patterns, such as canceling a request or providing
feedback during the execution of a request. \\
Parameters & Parameters are key-value pairs that can be used to
configure nodes and their behavior. \\
Launch files & A launch file is an XML file that specifies how to start
multiple nodes with their respective parameters and configurations. \\
RQT & RQT is a collection of ROS graphical tools for debugging,
monitoring, and visualizing ROS topics, nodes, and messages. \\
RViz & RViz is a 3D visualization tool for ROS that allows users to
display and interact with sensor data and robot models in real-time. \\
tf2 & tf2 is a library for keeping track of coordinate frames over time.
It provides a way to transform data between different coordinate frames
in a robot system. \\
ROS2 Middleware & ROS2 Middleware is the layer of software that provides
the communication infrastructure between nodes. ROS2 provides multiple
middleware options, such as Fast RTPS and Cyclone DDS. \\
\bottomrule()
\end{longtable}

\bookmarksetup{startatroot}

\hypertarget{creating-package}{%
\chapter{Creating package}\label{creating-package}}

\begin{tcolorbox}[enhanced jigsaw, coltitle=black, breakable, title=\textcolor{quarto-callout-important-color}{\faExclamation}\hspace{0.5em}{Important}, toprule=.15mm, leftrule=.75mm, colframe=quarto-callout-important-color-frame, colbacktitle=quarto-callout-important-color!10!white, opacitybacktitle=0.6, left=2mm, colback=white, toptitle=1mm, bottomtitle=1mm, titlerule=0mm, arc=.35mm, rightrule=.15mm, opacityback=0, bottomrule=.15mm]

You are reading the work-in-progress edition of ROS 2 : Mea Via et Mea
Disciplina. This chapter is currently a dumping ground for ideas, and we
don't recommend reading it. The complete version will be available here
soon. Stay tuned!

\end{tcolorbox}

\bookmarksetup{startatroot}

\hypertarget{section}{%
\chapter{}\label{section}}

\begin{tcolorbox}[enhanced jigsaw, coltitle=black, breakable, title=\textcolor{quarto-callout-important-color}{\faExclamation}\hspace{0.5em}{Important}, toprule=.15mm, leftrule=.75mm, colframe=quarto-callout-important-color-frame, colbacktitle=quarto-callout-important-color!10!white, opacitybacktitle=0.6, left=2mm, colback=white, toptitle=1mm, bottomtitle=1mm, titlerule=0mm, arc=.35mm, rightrule=.15mm, opacityback=0, bottomrule=.15mm]

You are reading the work-in-progress edition of ROS 2 : Mea Via et Mea
Disciplina. This chapter is currently a dumping ground for ideas, and we
don't recommend reading it. The complete version will be available here
soon. Stay tuned!

\end{tcolorbox}

\bookmarksetup{startatroot}

\hypertarget{section-1}{%
\chapter{}\label{section-1}}

\begin{tcolorbox}[enhanced jigsaw, coltitle=black, breakable, title=\textcolor{quarto-callout-important-color}{\faExclamation}\hspace{0.5em}{Important}, toprule=.15mm, leftrule=.75mm, colframe=quarto-callout-important-color-frame, colbacktitle=quarto-callout-important-color!10!white, opacitybacktitle=0.6, left=2mm, colback=white, toptitle=1mm, bottomtitle=1mm, titlerule=0mm, arc=.35mm, rightrule=.15mm, opacityback=0, bottomrule=.15mm]

You are reading the work-in-progress edition of ROS 2 : Mea Via et Mea
Disciplina. This chapter is currently a dumping ground for ideas, and we
don't recommend reading it. The complete version will be available here
soon. Stay tuned!

\end{tcolorbox}

\bookmarksetup{startatroot}

\hypertarget{section-2}{%
\chapter{}\label{section-2}}

\begin{tcolorbox}[enhanced jigsaw, coltitle=black, breakable, title=\textcolor{quarto-callout-important-color}{\faExclamation}\hspace{0.5em}{Important}, toprule=.15mm, leftrule=.75mm, colframe=quarto-callout-important-color-frame, colbacktitle=quarto-callout-important-color!10!white, opacitybacktitle=0.6, left=2mm, colback=white, toptitle=1mm, bottomtitle=1mm, titlerule=0mm, arc=.35mm, rightrule=.15mm, opacityback=0, bottomrule=.15mm]

You are reading the work-in-progress edition of ROS 2 : Mea Via et Mea
Disciplina. This chapter is currently a dumping ground for ideas, and we
don't recommend reading it. The complete version will be available here
soon. Stay tuned!

\end{tcolorbox}

\bookmarksetup{startatroot}

\hypertarget{section-3}{%
\chapter{}\label{section-3}}

\begin{tcolorbox}[enhanced jigsaw, coltitle=black, breakable, title=\textcolor{quarto-callout-important-color}{\faExclamation}\hspace{0.5em}{Important}, toprule=.15mm, leftrule=.75mm, colframe=quarto-callout-important-color-frame, colbacktitle=quarto-callout-important-color!10!white, opacitybacktitle=0.6, left=2mm, colback=white, toptitle=1mm, bottomtitle=1mm, titlerule=0mm, arc=.35mm, rightrule=.15mm, opacityback=0, bottomrule=.15mm]

You are reading the work-in-progress edition of ROS 2 : Mea Via et Mea
Disciplina. This chapter is currently a dumping ground for ideas, and we
don't recommend reading it. The complete version will be available here
soon. Stay tuned!

\end{tcolorbox}

\bookmarksetup{startatroot}

\hypertarget{section-4}{%
\chapter{}\label{section-4}}

\begin{tcolorbox}[enhanced jigsaw, coltitle=black, breakable, title=\textcolor{quarto-callout-important-color}{\faExclamation}\hspace{0.5em}{Important}, toprule=.15mm, leftrule=.75mm, colframe=quarto-callout-important-color-frame, colbacktitle=quarto-callout-important-color!10!white, opacitybacktitle=0.6, left=2mm, colback=white, toptitle=1mm, bottomtitle=1mm, titlerule=0mm, arc=.35mm, rightrule=.15mm, opacityback=0, bottomrule=.15mm]

You are reading the work-in-progress edition of ROS 2 : Mea Via et Mea
Disciplina. This chapter is currently a dumping ground for ideas, and we
don't recommend reading it. The complete version will be available here
soon. Stay tuned!

\end{tcolorbox}

\bookmarksetup{startatroot}

\hypertarget{section-5}{%
\chapter{}\label{section-5}}

\begin{tcolorbox}[enhanced jigsaw, coltitle=black, breakable, title=\textcolor{quarto-callout-important-color}{\faExclamation}\hspace{0.5em}{Important}, toprule=.15mm, leftrule=.75mm, colframe=quarto-callout-important-color-frame, colbacktitle=quarto-callout-important-color!10!white, opacitybacktitle=0.6, left=2mm, colback=white, toptitle=1mm, bottomtitle=1mm, titlerule=0mm, arc=.35mm, rightrule=.15mm, opacityback=0, bottomrule=.15mm]

You are reading the work-in-progress edition of ROS 2 : Mea Via et Mea
Disciplina. This chapter is currently a dumping ground for ideas, and we
don't recommend reading it. The complete version will be available here
soon. Stay tuned!

\end{tcolorbox}

\bookmarksetup{startatroot}

\hypertarget{section-6}{%
\chapter{}\label{section-6}}

\begin{tcolorbox}[enhanced jigsaw, coltitle=black, breakable, title=\textcolor{quarto-callout-important-color}{\faExclamation}\hspace{0.5em}{Important}, toprule=.15mm, leftrule=.75mm, colframe=quarto-callout-important-color-frame, colbacktitle=quarto-callout-important-color!10!white, opacitybacktitle=0.6, left=2mm, colback=white, toptitle=1mm, bottomtitle=1mm, titlerule=0mm, arc=.35mm, rightrule=.15mm, opacityback=0, bottomrule=.15mm]

You are reading the work-in-progress edition of ROS 2 : Mea Via et Mea
Disciplina. This chapter is currently a dumping ground for ideas, and we
don't recommend reading it. The complete version will be available here
soon. Stay tuned!

\end{tcolorbox}

\bookmarksetup{startatroot}

\hypertarget{section-7}{%
\chapter{}\label{section-7}}

\begin{tcolorbox}[enhanced jigsaw, coltitle=black, breakable, title=\textcolor{quarto-callout-important-color}{\faExclamation}\hspace{0.5em}{Important}, toprule=.15mm, leftrule=.75mm, colframe=quarto-callout-important-color-frame, colbacktitle=quarto-callout-important-color!10!white, opacitybacktitle=0.6, left=2mm, colback=white, toptitle=1mm, bottomtitle=1mm, titlerule=0mm, arc=.35mm, rightrule=.15mm, opacityback=0, bottomrule=.15mm]

You are reading the work-in-progress edition of ROS 2 : Mea Via et Mea
Disciplina. This chapter is currently a dumping ground for ideas, and we
don't recommend reading it. The complete version will be available here
soon. Stay tuned!

\end{tcolorbox}

\bookmarksetup{startatroot}

\hypertarget{section-8}{%
\chapter{}\label{section-8}}

\begin{tcolorbox}[enhanced jigsaw, coltitle=black, breakable, title=\textcolor{quarto-callout-important-color}{\faExclamation}\hspace{0.5em}{Important}, toprule=.15mm, leftrule=.75mm, colframe=quarto-callout-important-color-frame, colbacktitle=quarto-callout-important-color!10!white, opacitybacktitle=0.6, left=2mm, colback=white, toptitle=1mm, bottomtitle=1mm, titlerule=0mm, arc=.35mm, rightrule=.15mm, opacityback=0, bottomrule=.15mm]

You are reading the work-in-progress edition of ROS 2 : Mea Via et Mea
Disciplina. This chapter is currently a dumping ground for ideas, and we
don't recommend reading it. The complete version will be available here
soon. Stay tuned!

\end{tcolorbox}

\bookmarksetup{startatroot}

\hypertarget{section-9}{%
\chapter{}\label{section-9}}

\begin{tcolorbox}[enhanced jigsaw, coltitle=black, breakable, title=\textcolor{quarto-callout-important-color}{\faExclamation}\hspace{0.5em}{Important}, toprule=.15mm, leftrule=.75mm, colframe=quarto-callout-important-color-frame, colbacktitle=quarto-callout-important-color!10!white, opacitybacktitle=0.6, left=2mm, colback=white, toptitle=1mm, bottomtitle=1mm, titlerule=0mm, arc=.35mm, rightrule=.15mm, opacityback=0, bottomrule=.15mm]

You are reading the work-in-progress edition of ROS 2 : Mea Via et Mea
Disciplina. This chapter is currently a dumping ground for ideas, and we
don't recommend reading it. The complete version will be available here
soon. Stay tuned!

\end{tcolorbox}

\bookmarksetup{startatroot}

\hypertarget{section-10}{%
\chapter{}\label{section-10}}

\begin{tcolorbox}[enhanced jigsaw, coltitle=black, breakable, title=\textcolor{quarto-callout-important-color}{\faExclamation}\hspace{0.5em}{Important}, toprule=.15mm, leftrule=.75mm, colframe=quarto-callout-important-color-frame, colbacktitle=quarto-callout-important-color!10!white, opacitybacktitle=0.6, left=2mm, colback=white, toptitle=1mm, bottomtitle=1mm, titlerule=0mm, arc=.35mm, rightrule=.15mm, opacityback=0, bottomrule=.15mm]

You are reading the work-in-progress edition of ROS 2 : Mea Via et Mea
Disciplina. This chapter is currently a dumping ground for ideas, and we
don't recommend reading it. The complete version will be available here
soon. Stay tuned!

\end{tcolorbox}

\bookmarksetup{startatroot}

\hypertarget{section-11}{%
\chapter{}\label{section-11}}

\begin{tcolorbox}[enhanced jigsaw, coltitle=black, breakable, title=\textcolor{quarto-callout-important-color}{\faExclamation}\hspace{0.5em}{Important}, toprule=.15mm, leftrule=.75mm, colframe=quarto-callout-important-color-frame, colbacktitle=quarto-callout-important-color!10!white, opacitybacktitle=0.6, left=2mm, colback=white, toptitle=1mm, bottomtitle=1mm, titlerule=0mm, arc=.35mm, rightrule=.15mm, opacityback=0, bottomrule=.15mm]

You are reading the work-in-progress edition of ROS 2 : Mea Via et Mea
Disciplina. This chapter is currently a dumping ground for ideas, and we
don't recommend reading it. The complete version will be available here
soon. Stay tuned!

\end{tcolorbox}

\bookmarksetup{startatroot}

\hypertarget{section-12}{%
\chapter{}\label{section-12}}

\begin{tcolorbox}[enhanced jigsaw, coltitle=black, breakable, title=\textcolor{quarto-callout-important-color}{\faExclamation}\hspace{0.5em}{Important}, toprule=.15mm, leftrule=.75mm, colframe=quarto-callout-important-color-frame, colbacktitle=quarto-callout-important-color!10!white, opacitybacktitle=0.6, left=2mm, colback=white, toptitle=1mm, bottomtitle=1mm, titlerule=0mm, arc=.35mm, rightrule=.15mm, opacityback=0, bottomrule=.15mm]

You are reading the work-in-progress edition of ROS 2 : Mea Via et Mea
Disciplina. This chapter is currently a dumping ground for ideas, and we
don't recommend reading it. The complete version will be available here
soon. Stay tuned!

\end{tcolorbox}

\bookmarksetup{startatroot}

\hypertarget{section-13}{%
\chapter{}\label{section-13}}

\begin{tcolorbox}[enhanced jigsaw, coltitle=black, breakable, title=\textcolor{quarto-callout-important-color}{\faExclamation}\hspace{0.5em}{Important}, toprule=.15mm, leftrule=.75mm, colframe=quarto-callout-important-color-frame, colbacktitle=quarto-callout-important-color!10!white, opacitybacktitle=0.6, left=2mm, colback=white, toptitle=1mm, bottomtitle=1mm, titlerule=0mm, arc=.35mm, rightrule=.15mm, opacityback=0, bottomrule=.15mm]

You are reading the work-in-progress edition of ROS 2 : Mea Via et Mea
Disciplina. This chapter is currently a dumping ground for ideas, and we
don't recommend reading it. The complete version will be available here
soon. Stay tuned!

\end{tcolorbox}

\bookmarksetup{startatroot}

\hypertarget{section-14}{%
\chapter{}\label{section-14}}

\begin{tcolorbox}[enhanced jigsaw, coltitle=black, breakable, title=\textcolor{quarto-callout-important-color}{\faExclamation}\hspace{0.5em}{Important}, toprule=.15mm, leftrule=.75mm, colframe=quarto-callout-important-color-frame, colbacktitle=quarto-callout-important-color!10!white, opacitybacktitle=0.6, left=2mm, colback=white, toptitle=1mm, bottomtitle=1mm, titlerule=0mm, arc=.35mm, rightrule=.15mm, opacityback=0, bottomrule=.15mm]

You are reading the work-in-progress edition of ROS 2 : Mea Via et Mea
Disciplina. This chapter is currently a dumping ground for ideas, and we
don't recommend reading it. The complete version will be available here
soon. Stay tuned!

\end{tcolorbox}

\bookmarksetup{startatroot}

\hypertarget{section-15}{%
\chapter{}\label{section-15}}

\begin{tcolorbox}[enhanced jigsaw, coltitle=black, breakable, title=\textcolor{quarto-callout-important-color}{\faExclamation}\hspace{0.5em}{Important}, toprule=.15mm, leftrule=.75mm, colframe=quarto-callout-important-color-frame, colbacktitle=quarto-callout-important-color!10!white, opacitybacktitle=0.6, left=2mm, colback=white, toptitle=1mm, bottomtitle=1mm, titlerule=0mm, arc=.35mm, rightrule=.15mm, opacityback=0, bottomrule=.15mm]

You are reading the work-in-progress edition of ROS 2 : Mea Via et Mea
Disciplina. This chapter is currently a dumping ground for ideas, and we
don't recommend reading it. The complete version will be available here
soon. Stay tuned!

\end{tcolorbox}

\bookmarksetup{startatroot}

\hypertarget{section-16}{%
\chapter{}\label{section-16}}

\begin{tcolorbox}[enhanced jigsaw, coltitle=black, breakable, title=\textcolor{quarto-callout-important-color}{\faExclamation}\hspace{0.5em}{Important}, toprule=.15mm, leftrule=.75mm, colframe=quarto-callout-important-color-frame, colbacktitle=quarto-callout-important-color!10!white, opacitybacktitle=0.6, left=2mm, colback=white, toptitle=1mm, bottomtitle=1mm, titlerule=0mm, arc=.35mm, rightrule=.15mm, opacityback=0, bottomrule=.15mm]

You are reading the work-in-progress edition of ROS 2 : Mea Via et Mea
Disciplina. This chapter is currently a dumping ground for ideas, and we
don't recommend reading it. The complete version will be available here
soon. Stay tuned!

\end{tcolorbox}

\bookmarksetup{startatroot}

\hypertarget{section-17}{%
\chapter{}\label{section-17}}

\begin{tcolorbox}[enhanced jigsaw, coltitle=black, breakable, title=\textcolor{quarto-callout-important-color}{\faExclamation}\hspace{0.5em}{Important}, toprule=.15mm, leftrule=.75mm, colframe=quarto-callout-important-color-frame, colbacktitle=quarto-callout-important-color!10!white, opacitybacktitle=0.6, left=2mm, colback=white, toptitle=1mm, bottomtitle=1mm, titlerule=0mm, arc=.35mm, rightrule=.15mm, opacityback=0, bottomrule=.15mm]

You are reading the work-in-progress edition of ROS 2 : Mea Via et Mea
Disciplina. This chapter is currently a dumping ground for ideas, and we
don't recommend reading it. The complete version will be available here
soon. Stay tuned!

\end{tcolorbox}

\bookmarksetup{startatroot}

\hypertarget{section-18}{%
\chapter{}\label{section-18}}

\begin{tcolorbox}[enhanced jigsaw, coltitle=black, breakable, title=\textcolor{quarto-callout-important-color}{\faExclamation}\hspace{0.5em}{Important}, toprule=.15mm, leftrule=.75mm, colframe=quarto-callout-important-color-frame, colbacktitle=quarto-callout-important-color!10!white, opacitybacktitle=0.6, left=2mm, colback=white, toptitle=1mm, bottomtitle=1mm, titlerule=0mm, arc=.35mm, rightrule=.15mm, opacityback=0, bottomrule=.15mm]

You are reading the work-in-progress edition of ROS 2 : Mea Via et Mea
Disciplina. This chapter is currently a dumping ground for ideas, and we
don't recommend reading it. The complete version will be available here
soon. Stay tuned!

\end{tcolorbox}

\bookmarksetup{startatroot}

\hypertarget{section-19}{%
\chapter{}\label{section-19}}

\begin{tcolorbox}[enhanced jigsaw, coltitle=black, breakable, title=\textcolor{quarto-callout-important-color}{\faExclamation}\hspace{0.5em}{Important}, toprule=.15mm, leftrule=.75mm, colframe=quarto-callout-important-color-frame, colbacktitle=quarto-callout-important-color!10!white, opacitybacktitle=0.6, left=2mm, colback=white, toptitle=1mm, bottomtitle=1mm, titlerule=0mm, arc=.35mm, rightrule=.15mm, opacityback=0, bottomrule=.15mm]

You are reading the work-in-progress edition of ROS 2 : Mea Via et Mea
Disciplina. This chapter is currently a dumping ground for ideas, and we
don't recommend reading it. The complete version will be available here
soon. Stay tuned!

\end{tcolorbox}

\bookmarksetup{startatroot}

\hypertarget{section-20}{%
\chapter{}\label{section-20}}

\begin{tcolorbox}[enhanced jigsaw, coltitle=black, breakable, title=\textcolor{quarto-callout-important-color}{\faExclamation}\hspace{0.5em}{Important}, toprule=.15mm, leftrule=.75mm, colframe=quarto-callout-important-color-frame, colbacktitle=quarto-callout-important-color!10!white, opacitybacktitle=0.6, left=2mm, colback=white, toptitle=1mm, bottomtitle=1mm, titlerule=0mm, arc=.35mm, rightrule=.15mm, opacityback=0, bottomrule=.15mm]

You are reading the work-in-progress edition of ROS 2 : Mea Via et Mea
Disciplina. This chapter is currently a dumping ground for ideas, and we
don't recommend reading it. The complete version will be available here
soon. Stay tuned!

\end{tcolorbox}

\bookmarksetup{startatroot}

\hypertarget{section-21}{%
\chapter{}\label{section-21}}

\begin{tcolorbox}[enhanced jigsaw, coltitle=black, breakable, title=\textcolor{quarto-callout-important-color}{\faExclamation}\hspace{0.5em}{Important}, toprule=.15mm, leftrule=.75mm, colframe=quarto-callout-important-color-frame, colbacktitle=quarto-callout-important-color!10!white, opacitybacktitle=0.6, left=2mm, colback=white, toptitle=1mm, bottomtitle=1mm, titlerule=0mm, arc=.35mm, rightrule=.15mm, opacityback=0, bottomrule=.15mm]

You are reading the work-in-progress edition of ROS 2 : Mea Via et Mea
Disciplina. This chapter is currently a dumping ground for ideas, and we
don't recommend reading it. The complete version will be available here
soon. Stay tuned!

\end{tcolorbox}



\end{document}
